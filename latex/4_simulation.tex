\chapter{bubbleRob製作心得}
%\renewcommand{\baselinestretch}{10.0} %設定行距
\section{張昱棠心得}
我們在製作過程中遇到了相當多的問題,光語言的部分我們就開了一個翻譯的網頁在旁邊一起配著原文版看,在一開始時,有點不太理解調整數值是必須輸入註解內的數值還是本文內的,導致數值有些輸入錯誤,在一連串的錯誤後,我們果斷選擇,直接開一個新檔案重做,由於當天我們留在學校做,大概從七點開始一路錯誤重來錯誤重來到快凌晨一點才回家,終於在隔天成功做出結果來了,在學習coppeliasim 的部分,接續上學期學到的在這學期應用,並且學習了許多新的知識以及應用。
 \section{王翔楷心得}
 製作初期其實並不順利,我們這組和另外兩個同學一起從七點弄到凌晨一點才回家,過程中最難克服的是對軟體的不熟悉,隔天利用下午沒課的時間慢慢摸索,最後才順利完成建置,再利用晚上上課的時間向老師請教感測器內部程式的問題後回到座位上研究,最後才順利完成功課。
\newpage